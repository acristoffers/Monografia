% !TeX root = document.tex
% !TeX encoding = UTF-8 Unicode

\chapter{Revisão da literatura}%
\label{chp:bibliography-review}

A malha fechada foi formalizada matematicamente em 1868 por Maxwell. Em
1927~\textcite{article:black} demonstrou a utilidade da realimentação negativa
nos laboratórios da Bell onde aplicou a malha fechada com realimentação negativa
nos amplificadores das linhas de transmissão.

No entanto, ao ampliar o sinal, foi ampliado também o ruído. Para resolver este
problema iniciaram-se estudos para utilizar técnicas descritas por Laplace,
Fourier e Cauchy, que propunham o uso do domínio da frequência para modelar
sistemas dinâmicos. O problema passou a ser filtrar as frequências desejadas de
forma a não aplicar o ganho da malha fechada no rúido~\cite{book:bryson}.

Nyquist e~\textcite{article:bode} criaram ferramentas na década de 1930 que
permitem quantizar a estabilidade de um sistema em malha fechada. O uso de tais
ferramentas facilitou o desenvolvimento de controladores que alteram a dinâmica
do sistema para uma dinâmica desejada. A adição de ferramentas estocásticas
permitiu também o desenvolvimento de filtros ótimos.

Com estas técnicas estabeleceu-se o que chamamos de controle clássico. Este é
caracterizado pelo desenvolvimento no domínio da frequência e pelo uso de
técnicas projetadas para que seus cálculos fossem feitos à mão, ou no máximo com
o uso de tabelas~\cite{book:dorf}.

Embora o uso da análise no domínio da frequência tenha sido útil principalmente
para sua época, começaram a aparecer sistemas mais complexos, com várias
entradas e saídas, ordem elevada e não lineares, que não eram bem descritos no
domínio da frequência. Nesta época, estudos no domínio do tempo, utilizando as
equações diferenciais, começaram a aparecer, principalmente para resolver
problemas da indústria aeroespacial~\cite{article:lyapunov}.

Várias técnicas foram desenvolvidas nesta época. Uma se destaca: \ac{DMC} ---
\textit{Dynamic Matrix Control}. Ela visa resolver os problemas de controle
multivariável com restrições, típico nas indústrias química e petroquímica.
Antes do desenvolvimento desta técnica o controle era feito por várias malhas em
cascata~\cite{article:cutler}.

Seu impacto na indústria foi gigante. Provavelmente não há nenhuma empresa
extratora de petróleo que não utilize esta técnica ou uma derivada. O
desenvolvimento inical do \ac{MPC} se deu como uma tentativa de entender o
\ac{DMC}, que parecia desafiar a análise teórica tradicional por ser formulado
de maneira não convencional. Por exemplo, outra técnica, a \ac{IMC} ---
\textit{Internal Model Control} --- falhou em explicar o funcionamento da DMC
mas acabou ajudando no desenvolvimento do controle
robusto~\cite{article:morari}.

Hoje em dia o controle preditivo é formulado sempre em espaço de estados, mesmo
que seja possível sua formulação por, por exemplo, funções de transferência.
O controle preditivo por modelo é uma das formas de controle desenvolvidas com
a ideia de buscar a trajetória ótima de controle, que observa-se sempre existir
em sistemas dinâmicos. Para isso é utilizada a otimização custo, normalmente uma
função de energia~\cite{article:morari,book:bryson}.

O grande problema destes controladores hoje é o custo computacional quando se
necessita de um controlador \textit{on-line}, o que normalmente é o que se
deseja, pois estes retornam resultados melhores na presença de restrições. Os
estudos atuais seguem duas linhas: melhorar os \textit{solvers} de forma a
melhorar o desempenho ou encontrar maneiras de pré-processar as restrições, para
que apenas multiplicações matriciais sejam realizadas em tempo real. A primeira
linha de pesquisa é a mais explorada hoje em
dia~\cite{book:wang,masterthesis:zhang}.
