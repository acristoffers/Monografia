% !TeX root = document.tex
% !TeX encoding = UTF-8 Unicode

\chapter{Objetivos}%
\label{sec:objectives}

Desenvolver um controlador do tipo MPC com restrições com modelo SPD utilizando
observadores do tipo Kalman e implementá-lo utilizando e adaptando a plataforma
de controle desenvolvida pelo proponente para uso no forno do laboratório de
sinais e sistemas. O objetivo pode ser assim dividido:

\begin{itemize}
      \item modificar a eletrônica da planta: instalar circuitos
            microprocessados com o intuito de controlar o acionamento do forno,
            fornecendo um caminho alternativo ao acionamento que já está
            implementado e permitindo a integração com a plataforma de controle;
      \item modificar a plataforma de controle: desenvolver driver específico
            para o forno, de forma a tornar o acionamento dos atuadores e a
            leitura dos sensores mais simples e direto para os futuros usuários;
      \item desenvolver o controlador MPC\@: utilizar o modelo SPD desenvolvido
            pelo~\textcite{masterthesis:nelson} para desenvolver um
            controlador MPC com restrições na entrada e saída de forma a
            controlar a temperatura em um ponto diferente daquele onde está
            fisicamente instalado o sensor;
      \item implementar o controlador: utilizar a linguagem Python e a
            plataforma de controle para implementar o controlador e executar o
            controle da planta;
      \item comparar o desempenho do controlador: usar índices de desempenho
            para comparar o desempenho do controlador MPC com controladores PI a
            serem desenvolvidos utilizando as técnicas
            de~\textcite{article:clarke} e~\textcite{article:martins};
      \item realizar melhorias na plataforma: ao utilizar a plataforma como
            usuário, espera-se encontrar dificuldades, erros e novas ideias que
            serão ser corrigidos/implementadas, como, por exemplo, recuperação
            de mensagens de erro salvas no banco de dados e exibição para o
            usuário e verificação de sintaxe de código digitado pelo usuário.
\end{itemize}
