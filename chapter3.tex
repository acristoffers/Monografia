% !TeX root = document.tex
% !TeX encoding = UTF-8 Unicode

\chapter{Metodologia}%
\label{chp:methodology}

Para que a plataforma de controle possa se comunicar com o forno do laboratório
de sinais e sistemas foi necessário modificar o circuito de acionamento da
mesma. Para isso foram utilizados circuitos microcontrolados que se comunicam
com o circuito de acionamento e sensoriamento presente na mesma.

Esse novo circuito foi programado para se comunicar com a plataforma de
controle. Um \textit{driver} foi escrito para a mesma para permitir o controle
dos atuadores e leitura dos sensores de forma transparente e específica para
este sistema. O \textit{driver} abstrai a interface com o hardware de forma que
o usuário não precise se preocupar com detalhes da implementação, como número de
porta e pino, tensões de operação, etc.

Também foi necessário calibrar todos os sensores e atuadores, além de confererir
o funcionamento dos circuitos elétricos. Todo este procedimento foi feito
usando-se a plataforma de controle para a aquisição de dados. Para isso foi
necessário estudar os trabalhos de~\textcite{misc:nelson}
e~\textcite{misc:valle-silva}.

Os modelos levantados por~\textcite{masterthesis:nelson} foram validados no
sistema físico. Isto foi necessário para verificar que não houve alterações
significativas no sistema desde o seu desenvolvimento. Para o uso destes modelos
foi necessário estudar sobre sistemas a parâmetros distribuídos e observadores
de Kalman.

O controlador MPC utilizado é discreto no tempo e baseado em modelo em espaço de
estados, logo foi necessário estudar sobre controle digital e modelagem em
espaço de estados, além da própria técnica de controle. Para isto foram
utilizados, principalmente, os livros do~\textcite{book:dorf},
do~\textcite{book:ogata}, do~\textcite{book:wang} e as notas de aula do
Professor~\textcite{misc:patwardhan}.

Para o desenvolvimento do controlador foi utilizado o modelo SPD além de
determinadas restrições do sinal de controle. Os estados utilizados no
controlador são os provenientes do observador do tipo Kalman utilizado
pelo~\textcite{masterthesis:nelson}.

Para comparar o desempenho do controlador MPC foram desenvolvidos controladores
PI por síntese direta. Utilizou-se os índices \(IAE\) e \(IA\Delta{}X\) para
realizar uma comparação quantitativa dos controladores.

Todos os controladores foram implementados utilizando a plataforma de controle e
a linguagem de programação Python, assim como todos os testes que que usem o
acionamento do forno e/ou medição de seus termômetros. Desta forma foram
corrigidos na plataforma todos os problemas percebidos na mesma do ponto de
vista de usuário. Também foram implementadas várias novas funcionalidades,
visando seu aperfeiçoamento.
