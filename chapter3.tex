% !TeX root = document.tex
% !TeX encoding = UTF-8 Unicode

\chapter{Metodologia}%
\label{chp:methodology}

Para que a plataforma de controle possa se comunicar com o forno do laboratório
de sinais e sistemas será necessário modificar o circuito de acionamento da
mesma. Para isso serão utilizados circuitos microcontrolados que irão se
comunicar com o circuito de acionamento e sensoriamento presente na mesma. Um
estudo da forma como isso é feito hoje será realizado com a finalidade de
inserir um circuito paralelo ao atual, de forma que o usuário possa escolher
qual circuito de acionamento e aquisição de dados será utilizado.

Esse novo circuito será programado para se comunicar com a plataforma de
controle. Um \textit{driver} será escrito para a mesma para permitir o controle
dos atuadores e leitura dos sensores de forma transparente e específica para
este sistema. O \textit{driver} irá abstrair a interface com o hardware de forma
que o usuário não tenha que se preocupar com detalhes da implementação, como
número de porta e pino, tensões de operação, etc.

Para que se possa trabalhar com a planta também será necessário calibrar todos
os sensores e atuadores, além de se fazer uma conferência dos circuitos
elétricos, para certificar que estes ainda funcionam de forma correta. Todo este
procedimento será feito usando-se a plataforma de controle para a aquisição de
dados. Para isso será necessário estudar os trabalhos de~\textcite{misc:nelson}
e~\textcite{misc:valle-silva}. Como prova de conceito, também será utilizado um
computador portátil de baixo custo para o controle da planta.

Os modelos levantados por~\textcite{masterthesis:nelson} serão validados no
sistema físico. Isto é necessário para verificar que não houve alterações
significativas no sistema desde o seu desenvolvimento. Para o uso destes modelos
será necessário estudar sobre sistemas a parâmetros distribuídos e observadores
de Kalman.

O controlador MPC será discreto no tempo e baseado em modelo em espaço de
estados, logo será necessário estudar sobre controle digital e modelagem em
espaço de estados, além da própria técnica de controle. Para isto serão
utilizados, principalmente, os livros do~\textcite{book:dorf},
do~\textcite{book:ogata}, do~\textcite{book:wang} e as notas de aula do
Professor~\textcite{misc:patwardhan}.

Para o desenvolvimento do controlador será utilizado o modelo SPD e serão
determinadas restrições do sinal de controle. Os estados utilizados no
controlador serão os provenientes do observador do tipo Kalman utilizado
pelo~\textcite{masterthesis:nelson}.

Para comparar o desempenho do controlador MPC serão desenvolvidos controladores
PI utilizando as técnicas descritas por~\textcite{article:clarke}
e~\textcite{article:martins}. Serão utilizados os índices IAE e IVx para
realizar uma comparação quantitativa dos controladores.

Todos os controladores serão implementados utilizando a plataforma de controle e
a linguagem de programação Python, assim como todos os testes que requeiram
acionamento do forno e/ou medição de seus termômetros. Desta forma serão
corrigidos na plataforma todos os problemas percebidos na mesma do ponto de
vista de usuário. Também serão implementadas as funcionalidades que se mostrem
necessárias e/ou desejadas, visando seu aperfeiçoamento.
