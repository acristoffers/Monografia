% !TeX root = document.tex
% !TeX encoding = UTF-8 Unicode

\begin{abstract}
	Controle preditivo baseado em modelo (\ac{MPC} --- \textit{model predictive
	control}) é uma técnica avançada de controle que permite a inserção de
	restrições no sinal de controle e de variação dos estados do sistema na
	etapa de equacionamento do controlador. Seu objetivo é encontrar a
	trajetória de controle ótima que respeite as restrições impostas. O MPC já
	está estabelecido em indústrias que lidam com processos multivariáveis de
	dinâmica lenta, especialmente na indústria petro-química. Sua formulação
	mais comum utiliza modelos descrito no espaço de estados, o que requer que
	todos os estados sejam conhecidos. Como nem sempre é possível medi-los,
	faz-se necessário o uso de observadores, que são técnicas de estimar os
	estados do sistema a partir dos sinais de entrada e saída por meio do modelo
	dinâmico do sistema. Utilizando técnicas de modelagem, como por exemplo os
	métodos lineares \ac{ARMAX} (do inglês \textit{autoregressive moving avarage
	with exogenous inputs}), pode-se obter um modelo a parâmetros concentrados
	cujos coeficientes dependem do espaço (\ac{SPD} --- sistema a parâmetros
	distribuídos). Esses modelos SPD pode ser utilizados por observadores para
	estimar os estados do sistema em pontos de intenteresse onde não é possível
	ou viável inserir sensores. Assim pode-se, por exemplo, medir-se a
	temperatura na extremidade de um sólido e, através desta, recuperar-se a
	temperatura em algum ponto no meio do sólido, desde que o modelo dinâmico de
	propagação seja conhecido. Ao combinar o modelo \ac{ARMAX}, o controlador
	\ac{MPC} e o observador de estados é possível controlar uma variável em um
	ponto diferente daquele sendo medido. Assim propõe-se o desenvolvimento e
	implementação de um controlador \ac{MPC} que utilize modelos \ac{SPD} para
	realizar o controle de uma variável estimada em um ponto intermediário de um
	sistema distribuído espacialmente. Para isso será utilizada a planta
	presente no Laboratório de Sinais e Sistemas do \textit{campus} V do
	CEFET-MG. O modelo \ac{SPD} utilizado é o desenvolvido
	por~\textcite{masterthesis:nelson}. A implementação é feita utilizando os
	softwares \textit{Moirai} e \textit{Lachesis}, desenvolvidos \textit{in
	loco}, que foram atualizados de forma a comunicar com o \textit{hardware} da
	planta que receberam atualizações visando suas melhorias. Pretende-se com
	este trabalho, que envolve principalmente as áreas de controle e computação,
	aprofundar os estudos do Grupo de Modelagem e Controle de Sistemas
	Mecatrônicos na utilização de modelos \ac{SPD} e do \ac{MPC},  bem como
	facilitar futuros trabalhos nesta e outras plantas.
\end{abstract}

Palavras-chave: Controle preditivo por modelo, sistema a parâmetros
                distribuídos, observador de Kalman

\cleardoublepage{}

\begin{otherlanguage}{english}
	\begin{abstract}
		Model Predictive Control (\ac{MPC}) is an advanced control technique
		that allows the insertion of restrictions in the control signal and in
		the variation of the system states in the controller equation stage. Its
		goal is to find the optimal trajectory of the control signal that
		respects the restrictions imposed. The MPC is already established in
		industries dealing with multivariate, slow dynamics systems, especially
		in the petrochemical industry. Its formulation most commonly uses models
		described in state space, what requires all states to be known. As it is
		not always possible to measure them, observers, which are techniques of
		estimating states from the input and output signals, are employed. Using
		modeling techniques, such as \ac{ARMAX} (autoregressive moving avarage
		with exogenous inputs), one can obtain a model with concentrated
		parameters whose coefficients depend on the space (DPS --- distributed
		parameters system). These DPS models can be used to estimate the states
		of the system at points of interest where it is not possible or feasible
		to insert sensors. Thus, one can measure, for example, the temperature
		at the end of a solid and, through an observer, recover the temperature
		at some point in the middle of the solid, provided that the dynamic
		propagation is known. When combining the \ac{ARMAX} model, the \ac{MPC}
		controller and the state observer it is possible to control a variable
		in a different spacial point from that being measured. Thus, the
		development and implementation of a \ac{MPC} controller using DPS to
		control an estimated variable at an intermediate point of a spatially
		distributed system is proposed. For this the plant used will be the one
		present in the Laboratory of Signals and Systems at \textit{campus} V of
		CEFET-MG. The DPS model used is the one developed
		by~\textcite{masterthesis:nelson}. The implementation is done using the
		Moirai and Lachesis softwares, developed \textit{in loco}, which have
		been updated in order to communicate with the hardware of the plant. The
		software received updates aimed at its improvements. It is intended to
		this work, which mainly involves the areas of control and computation,
		to deepen the studies of the Group of Modeling and Control of Systems
		Mechatronics in the use of DPS and MPC models, as well as facilitate
		future work in this and other plants.
    \end{abstract}
    
    Keywords: Model predictive control, distributed parameters system, Kalman
              filter
\end{otherlanguage}
