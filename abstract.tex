% !TeX root = document.tex
% !TeX encoding = UTF-8 Unicode

\chapter*{Resumo}

\begin{abstract}
    Observadores permitem que o estado do sistema seja estimado dada sua saída.
    Sistemas a parâmetros distribuídos (SPD) podem ser utilizados com
    observadores para recuperar informações ao longo do processo. Assim pode-se,
    por exemplo, medir-se a temperatura na extremidade de um sólido e, através
    desta, recuperar-se a temperatura em algum ponto no meio do sólido. Controle
    preditivo baseado em modelo (MPC --- \textit{model predictive control}) é
    uma técnica avançada de controle que lida com restrições. Ela já está
    estabelecida em indústrias que lidam com processos multi-variáveis de
    dinâmica lenta, especialmente na indústria petro-química. Ao combinar as
    duas técnicas é possível não apenas controlar uma variável em um ponto
    diferente daquele sendo medida, como também usar esta como uma restrição
    física no controle de outro ponto. Isto permite o controle de variáveis que
    são difíceis ou inviáveis de serem medidas diretamente. Embora ambas
    técnicas sejam bem discutidas isoladamente, há poucos artigos onde as duas
    são usadas em conjunto, o que sugere a existência de um campo a ser
    explorado. Assim propõe-se o desenvolvimento e a implementação de um
    controlador MPC que utilize SPD para realizar o controle de uma variável
    estimada em um ponto intermediário. Para isso será utilizada a planta
    presente no laboratório de sinais e sistemas. O modelo SPD utilizado será o
    desenvolvido por~\textcite{masterthesis:nelson}. A implementação será feita
    utilizando os softwares \textit{Moirai} e \textit{Lachesis}, desenvolvidos
    \textit{in loco}, o que irá requerer modificações nos mesmos para
    funcionarem com as especificidades da planta. Também será feita alteração na
    microeletrônica da planta, com a finalidade de prover um meio alternativo
    para acionamento do forno. As modificações de hardware e software, em
    conjunto, permitirão maior flexibilidade no uso da planta, além de maior
    facilidade de uso. Pretende-se com este trabalho, que envolve principalmente
    as áreas de controle e computação, aprofundar os estudos do grupo de
    modelagem e controle de processos em SPD com a metodologia de controle
    proposta, bem como facilitar futuros trabalhos na planta utilizada através
    do melhor desenvolvimento e teste da plataforma de controle.
\end{abstract}

Palavras-chave: Controle preditivo por modelo, sistema a parâmetros
                distribuídos, observador de Kalman
