% !TeX root = document.tex
% !TeX encoding = UTF-8 Unicode

\begin{abstract}
	Controle preditivo baseado em modelo (\ac{MPC} --- \textit{model predictive
	control}) é uma técnica avançada de controle que permite a inserção de
	restrições no equacionamento do controlador. Seu objetivo é encontrar a
	trajetória de controle ótima que respeite as restrições impostas, que podem
	ser na amplitude ou variação do sinal de controle, de saída ou de um estado.
	Ele já está estabelecida em indústrias que lidam com processos
	multi-variáveis de dinâmica lenta, especialmente na indústria petro-química.
	Sua formulação mais comum é utilizando modelos por espaço de estados, o que
	requer que todos os estados sejam conhecidos. Como nem sempre é possível
	medi-los, faz-se necessário o uso de observadores, que são técnicas de
	estimar os estados do sistema a partir do modelo, sinal de controle e um
	estado medido. Utilizando técnicas de modelagem pouco exploradas, como
	sistemas a parâmetros distribuídos (\ac{SPD}), pode-se utilizar observadores
	para estimar parâmetros do sistema em pontos de intenteresse onde não é
	possível ou viável inserir sensores. Assim pode-se, por exemplo, medir-se a
	temperatura na extremidade de um sólido e, através desta, recuperar-se a
	temperatura em algum ponto no meio do sólido. Ao combinar as três técnicas é
	possível controlar uma variável em um ponto diferente daquele sendo medido.
	Assim propõe-se o desenvolvimento e implementação de um controlador \ac{MPC}
	que utilize \ac{SPD} para realizar o controle de uma variável estimada em um
	ponto intermediário. Para isso será utilizada a planta presente no
	laboratório de sinais e sistemas. O modelo \ac{SPD} utilizado será o
	desenvolvido por~\textcite{masterthesis:nelson}. A implementação será feita
	utilizando os softwares \textit{Moirai} e \textit{Lachesis}, desenvolvidos
	\textit{in loco}, que serão atualizados de forma a comunicar com o
	\textit{hardware} da planta e receberá atualizações visando sua melhoria.
	Também será feita alteração na microeletrônica da planta, com a finalidade
	de prover um meio alternativo para acionamento do forno. Pretende-se com
	este trabalho, que envolve principalmente as áreas de controle e computação,
	aprofundar os estudos do grupo de modelagem e controle de processos
	mecatrônicos em \ac{SPD} e \ac{MPC}, bem como facilitar futuros trabalhos
	nesta e outras plantas, através do melhor desenvolvimento e teste da
	plataforma de controle.
\end{abstract}

Palavras-chave: Controle preditivo por modelo, sistema a parâmetros
                distribuídos, observador de Kalman

\cleardoublepage{}

\begin{otherlanguage}{english}
	\begin{abstract}
		Model predictive control (MPC) is an advanced control technique that
		allows the insertion of constraints in the controller equation. Its
		objective is to find the optimal control trajectory that respects the
		constraints imposed, which can be in the amplitude or variation of the
		control signal, output or a state. It is already established in
		industries dealing with multi-variable processes of slow dynamics,
		especially in the petrochemical industry. Its most common formulation is
		to use state-space models, which requires all states to be known. As it
		is not always possible to measure them, it is necessary to use
		observers, which are techniques of estimating the states of the system
		from the model, control signal and a measured state. Using less
		exploited modeling techniques, such as distributed parameters systems
		(DPS), observers can be used to estimate system parameters at points of
		intent where it is not possible or feasible to insert sensors. Thus, for
		example, one can measure the temperature at the end of a solid and
		therethrough recover the temperature at some point in the middle of the
		solid. By combining the three techniques it is possible to control a
		variable at a point other than the one being measured. Thus it is
		proposed the development and implementation of an MPC controller that
		uses SPD to perform the control of an estimated variable in an
		intermediate point. For this, the plant will be used in the signal and
		system laboratory. The SPD model used will be the one developed
		by~\textcite{masterthesis:nelson}. The implementation will be made using
		the Moirai and Lachesis softwares, developed \textit{in loco}, which
		will be updated in order to communicate with the plant hardware and
		receive updates to improve it. Also, there will be made change in the
		microelectronics of the plant, in order to provide an alternative means
		to drive it. It is intended that this work, which mainly involves the
		areas of control and computation, to deepen the studies of the modeling
		and control group of mechatronic processes in DPS and MPC, as well as to
		facilitate future works in this and other plants, through the best
		development and test of the control platform.
    \end{abstract}
    
    Keywords: Model predictive control, distributed parameters system, Kalman
              filter
\end{otherlanguage}
