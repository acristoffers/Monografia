% !TeX root = document.tex
% !TeX encoding = UTF-8 Unicode

\chapter{Introdução}%
\label{chp:introducao}

Controle preditivo por modelo (\ac{MPC} --- \textit{Model Predictive Control}) é
uma técnica de controle avançada que utiliza o modelo do sistema para prever a
saída em momentos futuros e, com isso, gerar uma estratégia de controle que
otimize algum critério quadrático selecionado. Esta técnica pode ser aplicada em
modelos contínuos quanto discretos. A abordagem discreta no tempo é usada neste
trabalho, pois melhor ilustra o funcionamento do \ac{MPC}, além de fornecer ao
controlador mais tempo para o processamento do próximo sinal de controle. Note
que a restrição de tempo para processamento do MPC pode ser crítica em algumas
aplicações, pois um problema de otimização deve ser resolvido a cada iteração.

Em um modelo discreto o sinal é amostrado a cada \( \Delta{}t \) segundos. O
modelo não varia mais diretamente com o tempo, mas sim com a amostragem \( k \),
que se associa com o tempo como \( t = k\Delta{}t \). Assim, em um modelo
discreto, ao se passar do tempo \( 0 \) para o tempo \(\Delta{}t \) ou do tempo
\( \Delta{}t \) para o tempo \( 2\Delta{}t \), dizemos que se passou um
instante.

Utilizando um modelo discreto pode-se então prever a saída do sistema para um
instante futuro, se as entradas que foram aplicadas ao sistema nos instantes
anteriores forem conhecidas. Pode-se então prever as saídas dos próximos \(N_p\)
instantes (chamado horizonte de predição) e utilizar técnicas de otimização para
obtermos os próximos \(N_c\) sinais de controle (chamado horizonte de controle)
ótimos que levarão o sistema à referência. Ao utilizar apenas o primeiro sinal
de controle e recalcular os sinais de controle ótimos a todo instante, o
controlador será capaz de rejeitar ruídos e reagir a mudanças, como a da
referência. Esta técnica é chamada de Controle por Horizonte Recessivo e é a
base do \ac{MPC}~\cite{book:wang}.

A discretização foi feita no tempo, pois altera-se o modelo para que esse não
mais dependa do tempo mas sim de um instante. Sistemas que dependem apenas do
tempo são chamados de sistemas a parâmetros concentrados (\ac{SPC}). No entanto
existem sistemas que não variam apenas no tempo, mas também no espaço. Um
exemplo é a transferência de calor em uma barra metálica, em que a temperatura
depende tanto do tempo quanto do ponto no espaço em que se faz a medição.
Sistemas que dependem de uma variável espacial são chamados de sistemas a
parâmetros distribuídos (\ac{SPD}). Sua modelagem é feita utilizando derivadas
parciais e equações diferenciais parciais (\ac{EDP})~\cite{phdthesis:caldeira}.

Para ambos tipos de modelos é comum usar a modelagem por espaço de estados ao se
trabalhar com \ac{MPC}. Isso permite projetar mais facilmente sistemas com
múltiplas entradas e múltiplas saídas. No entanto, ao se trabalhar com espaço de
estados, é necessário que o valor de todos os estados sejam conhecidos, o que
nem sempre é possível, seja por limitações do econômicas ou até mesmo físicas.
Para resolver esse problema utilizam-se observadores.

O observador é um sistema que estima os estados de um modelo baseado nas
entradas e saídas do mesmo. O observador de Kalman, também conhecido como filtro
de Kalman, é uma formulação que permite estimar sinais de forma ótima na
presença na presença de ruídos, seja na leitura da saída ou dos próprios
estados~\cite{book:wang}.

\section{Definição do Problema}%
\label{sec:definicao-do-problema}

A Figura~\ref{fig:sensors-SPD} mostra os sensores do forno presente no
Laboratório de Sinais e Sistemas. Nela podemos ver os sensores \(S_1\) a \(S_5\)
e o fluxo de ar quente \( q \). Imagine que queiramos controlar a temperatura na
posição onde encontra-se o sensor \(S_3\), mas que apenas o sensor \(S_5\)
esteja funcionando. Com um modelo \ac{SPD} podemos ter dois modelos \ac{SPC}\@:
um que modele o fluxo de temperatura a partir do atuador até o sensor \(S_3\) e
um outro que modele de \(S_3\) até \(S_5\), ambos interconectados. Desta forma,
podemos utilizar um observador e a leitura do sensor \(S_5\) para estimar a
temperatura no sensor \(S_3\), e controlá-la mesmo sem fazer sua medição direta.

\begin{figure}[ht!]
    \centering
    \captionsetup{justification=centering}
    \includegraphics[height=0.5\linewidth]{imgs/planta-spd}
    \caption{Sensores da planta}%
    \label{fig:sensors-SPD}
\end{figure}

\section{Objetivos}%
\label{sec:objectives}

O objetivo desse trabalho é desenvolver um controlador do tipo \ac{MPC} com
restrições na amplitude do sinal de controle e estados do sistema com modelo
\ac{SPD} utilizando observadores do tipo Kalman e implementá-lo utilizando e
adaptando a plataforma de controle desenvolvida pelo proponente para uso no
forno do laboratório de sinais e sistemas. Para alcançar esse objetivo as
seguintes ações foram realizadas:

\begin{itemize}
      \item modificar a eletrônica da planta: instalar circuitos
            microprocessados com o intuito de controlar o acionamento do forno,
            fornecendo um caminho alternativo ao acionamento que já está
            implementado e permitindo a integração com a plataforma de controle;
      \item modificar a plataforma de controle: desenvolver driver específico
            para o forno, de forma a tornar o acionamento dos atuadores e a
            leitura dos sensores mais simples e direto para os futuros usuários;
      \item desenvolver o controlador \ac{MPC}\@: utilizar o modelo \ac{SPD}
            desenvolvido por~\textcite{masterthesis:nelson} para desenvolver um
            controlador \ac{MPC} com restrições na entrada e saída de forma a
            controlar a temperatura em um ponto diferente daquele onde está
            fisicamente instalado o sensor;
      \item implementar o controlador: utilizar a linguagem Python e a
            plataforma de controle para implementar o controlador e executar o
            controle da planta;
      \item comparar o desempenho do controlador: usar índices de desempenho
            para comparar o desempenho do controlador \ac{MPC} com controlador
            PI ou PID, desenvolvido utilizando síntese direta;
      \item realizar melhorias na plataforma: ao utilizar a plataforma como
            usuário, espera-se encontrar dificuldades, erros e novas ideias que
            serão corrigidos/implementadas, como, por exemplo, recuperação de
            mensagens de erro salvas no banco de dados e exibição para o usuário
            e verificação de sintaxe de código digitado pelo usuário.
\end{itemize}

\section{Motivação}%
\label{sec:motivacao}

A principal motivação para o trabalho é poder fazer o controle de um processo
sem medir diretamente a variável controlada. Isto é interessante pois nem sempre
é possível medir diretamente a grandeza que se deseja controlar, seja por
restrições físicas (como, por exemplo, colocar um sensor no centro de um
alto-forno) ou por questões financeiras.

A escolha do controlador \ac{MPC} se tornou interessante por ser uma técnica
avançada de controle que ainda não é amplamente estudada no \textit{campus V}.
Assim, além da possibilidade de estudar uma técnica avançada, este trabalho
contribui por ser o primeiro trabalho a utilizar tal controlador no Grupo de
Modelagem e Controle de Sistemas Mecatrônicos.

\section{Estado da arte}%
\label{sec:estado-da-arte}

O uso de controladores \ac{MPC} com modelos \ac{SPD} não é novo. Sua maior
aplicação, no entanto, continua sendo na área da química, fabricação de aço e,
principalmente na indústria petroquímica. Ele foi desenvolvido inicialmente para
controle de processos químicos em que o equipamento é caro, o processo é lento
e contém várias entradas, saídas e estados que devem ser restringidas em
amplitude e taxa de variação~\cite{article:cairano}.

Este cenário está mudando e outras indústrias --- com modelos com poucas
entradas e saídas, custo mais baixo e dinâmica mais rápida --- estão adotodando
os controladores \ac{MPC}\@. No entanto, isso apresenta alguns desafios no
desenvolvimento de controladores, pois o \textit{framework} \ac{MPC} não foi
desenvolvido para tais sistemas e possui limitações como, por exemplo, o alto
custo computacional. Trabalhos já estão sendo desenvolvidos para minimizar ou
sanar tais problemas~\cite{article:cairano}.

Em ambientes industriais é comum o uso de computadores lógicos programáveis para
fazer a interface com o hardware. No meio acadêmico, no entanto, utiliza-se
softwares como o \textit{MATLAB} e \textit{LabVIEW} para a prototipagem rápida.
Isto requer que o pesquisador recrie a interface gráfica e controle a aquisição
de dados manualmente, o que consome tempo que poderia ser gasto com a pesquisa.

Assim, uma plataforma que faça a interface com o hardware e possibilite a
execução de um controlador de forma transparente permite que o pesquisador se
concentre nestes estudos e não se preocupe com detalhes da implementação da
aquisição dos dados. Tal ferramenta também permitiria ao pesquisador
interagir com CLPs, desenvolvendo e testando controladores de forma fácil em
uma linguagem simples (Python) sem se preocupar com todas as nuâncias das
linguagens LADDER utilizadas nos CLPs, podendo implementar seu controlador nesta
linguagem limitada apenas ao final, quando este tiver sido devidamente testado e
otimizado.

A aplicação de uma plataforma para controle em todos as plantas do laboratório
permite que os usuários possam migrar de uma planta para outra sem dificuldades.
Também permite que controladores e observadores desenvolvidos sejam testados em
diferentes plantas com poucas modificações, já que as interfaces seriam as
mesmas em todas elas.
