% !TeX root = document.tex
% !TeX encoding = UTF-8 Unicode

\chapter{Recursos Necessários}%
\label{sec:needed-resources}

Para o novo circuito de acionamento, que será inserido paralelo ao existente,
serão necessários circuitos microcontrolados capazes de medir os valores de
tensão existentes no circuito atual e de se comunicar com um computador. Por
isso optou-se por utilizar placas Arduíno, pois são acessíveis e atendem aos
requisitos.

Também optou-se por utilizar um computador portátil de baixo custo como prova de
conceito, pois a plataforma de controle foi pensada de forma a executar bem o
controlador em um computador com recursos limitados. Para isto escolheu-se
utilizar um \textit{Raspberry Pi}. A comunicação entre o \textit{Raspberry} e os
Arduínos se dará por cabo USB (porta serial emulada), enquanto a comunicação do
\textit{Raspberry} com o exterior se dará por cabo de rede.

Com isso, necessitasse dos seguintes componentes:

\begin{itemize}
      \item 1 Raspberry Pi
      \item 2 Arduínos
      \item \textit{jumpers}
      \item Cabos USB
      \item Cabo de rede
\end{itemize}

Todo software utilizado será \textit{open-source}, como: linguagem Python,
plataforma de controle (Lachesis e Moirai), sistema operacional Raspbian, banco
de dados Mongo DB, ambiente de virtualização Docker, suítes SciPy e bibilioteca
Control.
