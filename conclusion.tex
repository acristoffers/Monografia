% !TeX root = document.tex
% !TeX encoding = UTF-8 Unicode

\chapter{Cronograma}%
\label{sec:chronogram}

A seguir está apresentado o cronograma de atividades a serem executadas.

\renewcommand{\labelenumii}{\theenumii}
\renewcommand{\theenumii}{\theenumi.\arabic{enumii}.}

\begin{enumerate}
      \item Estudos teóricos
            \begin{enumerate}
                  \item MPC --- será estudado o desenvolvimento e implementação
                        do controlador preditivo discreto no tempo com
                        restrições;
                  \item SPD --- será estudado o trabalho
                        do~\textcite{masterthesis:nelson}, visando replicar
                        seus resultados na plataforma;
                  \item Modelagem térmica --- serão estudados os trabalhos de
                        conclusão de curso do~\textcite{misc:nelson}
                        e~\textcite{misc:valle-silva}, que fizeram a
                        construção física e modelagem do forno;
            \end{enumerate}
      \item Modificação do hardware e implantação da plataforma
            \begin{enumerate}
                  \item Implementação de um circuito de acionamento do forno
                        paralelo ao existente atualmente;
                  \item Implantação de um Raspberry Pi para comunicação com os
                        Arduínos;
                  \item Instalação da plataforma no Raspberry Pi e modificação
                        da plataforma e do Arduíno para possibilitar a
                        comunicação e controle da planta;
            \end{enumerate}
      \item Calibrações, testes e certificações
            \begin{enumerate}
                  \item Certificação do funcionamento da parte elétrica do
                        forno;
                  \item Calibração dos sensores e atuadores;
                  \item Validação dos modelos SPD\@;
            \end{enumerate}
      \item Desenvolvimento do controlador
            \begin{enumerate}
                  \item Desenvolvimento de um controlador MPC discreto no tempo
                        com restrições baseado no modelo SPD com atraso nos
                        estados;
                  \item Desenvolvimento de um controlador PI para comparação;
                  \item Implementação do controlador na plataforma utilizando a
                        linguagem Python;
                  \item Comparação dos controladores MPC e PI utilizando os
                        índices IAE e IVx;
                  \item Aplicação de melhorias na plataforma de controle com
                        base na experiência adquirida durante seu uso;
            \end{enumerate}
      \item Relatório
            \begin{enumerate}
                  \item Escrita do relatório final de TCC1.
                  \item Defesa do TCC1
                  \item Escrita do relatório final de TCC2.
                  \item Defesa do TCC2
            \end{enumerate}
\end{enumerate}

\begin{table}[!ht]
      \caption{Cronograma de atividades}
      \begin{center}
            \footnotesize{
            \begin{tabular}{c|c|c|c|c|c|c|c|c|c}
                \hline
                Atividade ($\downarrow$) Mês.($\rightarrow$) & Mar     & Abr     & Mai     & Jun     & Jul     & Ago     & Set     & Out     & Nov     \\ \hline
                $1.1$                                        & $\surd$ & $\surd$ & $\surd$ & $\surd$ & $\surd$ & $\surd$ & $\surd$ & $\surd$ &         \\ \hline
                $1.2$                                        & $\surd$ & $\surd$ &         &         &         &         &         &         &         \\ \hline
                $1.3$                                        & $\surd$ & $\surd$ &         &         &         &         &         &         &         \\ \hline
                $2.1$                                        & $\surd$ & $\surd$ &         &         &         &         &         &         &         \\ \hline
                $2.2$                                        & $\surd$ & $\surd$ &         &         &         &         &         &         &         \\ \hline
                $2.3$                                        & $\surd$ & $\surd$ &         &         &         &         &         &         &         \\ \hline
                $3.1$                                        &         & $\surd$ & $\surd$ &         &         &         &         &         &         \\ \hline
                $3.2$                                        &         & $\surd$ & $\surd$ &         &         &         &         &         &         \\ \hline
                $3.3$                                        &         & $\surd$ & $\surd$ &         &         &         &         &         &         \\ \hline
                $4.1$                                        &         &         &         & $\surd$ & $\surd$ & $\surd$ &         &         &         \\ \hline
                $4.2$                                        &         &         &         & $\surd$ & $\surd$ & $\surd$ &         &         &         \\ \hline
                $4.3$                                        &         &         &         & $\surd$ & $\surd$ & $\surd$ &         &         &         \\ \hline
                $4.4$                                        &         &         &         &         &         & $\surd$ &         &         &         \\ \hline
                $4.5$                                        &         &         &         & $\surd$ & $\surd$ & $\surd$ &         &         &         \\ \hline
                $5.1$                                        & $\surd$ & $\surd$ & $\surd$ & $\surd$ &         &         &         &         &         \\ \hline
                $5.2$                                        &         &         &         & $\surd$ &         &         &         &         &         \\ \hline
                $5.3$                                        &         &         &         & $\surd$ & $\surd$ & $\surd$ & $\surd$ & $\surd$ &         \\ \hline
                $5.4$                                        &         &         &         &         &         &         &         &         & $\surd$ \\ \hline
            \end{tabular}
        }
      \end{center}
\end{table}
