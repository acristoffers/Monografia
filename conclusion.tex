% !TeX root = document.tex
% !TeX encoding = UTF-8 Unicode

\chapter{Considerações finais}%
\label{chp:conclusion}

Os controladores PI e MPC apresentaram respostas parecidas. No entanto, o
controlador MPC obteve uma resposta até 27\% melhor segundo o índice IVu, o que
se traduz em economia de energia por menos mudanças no atuador. Já o ITAE mostra
o MPC com uma vantagem de até 15\% sobre o PI.\@

Estudos futuros poderiam utilizar outro sistema para melhor explorar os pontos
chaves do MPC, que são a saturação de atuadores, sinal de saída e variação do
sinal de controle, bem como o uso de múltiplas entradas e múltiplas saídas e de
ordem elevadas sem modificação do \textit{framework} do controlador. Esses casos
não foram abordados pois o modelo de interesse, que é SISO de segunda ordem, foi
linearizado longe da saturação do atuador.

A modelagem do sistema a parâmetros distribuídos via subsistemas interconectados
possibilitou o controle da temperatura em um ponto onde não há sensor. Foi
detectada a necessidade de melhorar o modelo, inserindo a temperatura ambiente
como parâmetro de entrada, mas isso não foi impedimento para a realização do
controle. A alternativa utilizada não se mostrou viável e recomenda-se que o
modelo seja melhorado. Uma alternativa baseada em observadores também pode ser
estudada.

Os observadores não funcionaram conforme esperado. Independente dos valores
ajustáveis (\(R\) e \(Q\)) o estado estimado sempre converge para um valor e não
mais varia junto com o estado medido, a não ser que a variação desse seja
grande. O efeito é aquele de um filtro, mas ignorando a matriz de covariança.
Assim, para trabalhos futuros, recomenda-se o estudo de desenvolvimento e
implementação de observadores, visando, principalmente, resolver o problema
encontrado no modelo.

A partir do trabalho aqui desenvolvido foi publicado um artigo na revista romena
\textit{Studies in Informatics and Control}, volume 27, número 3, de setembro de
2018 entitulado \href{https://doi.org/10.24846/v27i3y201802}{\textit{Affordable
Control Platform with MPC Application}} (Plataforma de Controle de Baixo Custo
com aplicação de MPC). O trabalho envolve duas das principais áreas desse
trabalho, o controlador MPC e a plataforma de controle. O resumo do trabalho é
apresentado, traduzido, a seguir.

\begin{quote}
    Este artigo apresenta uma plataforma de controle desenvolvida para
    interfacear com vários hardwares, permitindo o design e a rápida
    implementação até mesmo de controladores avançados, tanto em sistemas
    acadêmicos quanto industriais. O código dos controladores é escrito na
    linguagem Python, de código aberto, facilitando a tradução de código
    normalmente escrito em software comercial. A plataforma proposta pode usar
    desde Arduinos até Computadores Lógicos Programáveis ​​(PLCs). Além da
    pesquisa e testes em instalações industriais, a simplicidade da plataforma
    proposta permite seu uso também para fins educacionais e de treinamento.
    Portanto, a plataforma proposta pode ajudar os alunos a se concentrarem na
    análise de sistemas e na teoria de controle, em vez de problemas de
    interface de hardware, enquanto usam hardware de baixo custo. Desenvolvida
    em um esquema cliente-servidor, a plataforma pode ser executada em
    computadores acessíveis, aproveitando as ferramentas matemáticas e gráficas
    de alto nível disponíveis na linguagem Python, permitindo a rápida
    implementação de controladores avançados. O uso desta plataforma é ilustrado
    com a implementação de um controle preditivo modelo (MPC) de um controle de
    nível em um processo em escala de laboratório. Um PLC é usado para tomar as
    medidas de nível, para despachar sinais de controle e também para
    intertravar tarefas seguras. O controlador é executado em um computador
    Raspberry Pi que se comunica com o PLC por meio de um link Ethernet.
\end{quote}
